\begin{abstract}
    \noindent
    The HealthFlow Connect project aims to digitize the case paper system used in government hospitals in India. This project leverages the MERN (MongoDB, Express.js, React.js, Node.js) stack to develop a comprehensive web-based platform for managing patient information and medical records efficiently. The system caters to various user roles including admin, doctor, counter staff, lab technician, pharmacist, and patients, each with specific functionalities tailored to their roles.
    
    The admin, typically a doctor overseeing hospital administration, has the responsibility of managing user profiles, including doctors, pharmacists, counter staff, and lab technicians. They have overall access to user information and system settings. Doctors utilize the platform to diagnose patients, record patient complaints, conduct examinations, prescribe medications, and maintain comprehensive patient histories for effective diagnosis and treatment planning. Counter staff handle patient registration, record updates, case paper creation, and printing, ensuring seamless patient management.
    
    Lab technicians manage assigned lab tests and upload test reports, while pharmacists oversee medication requirements and dispensation, all of which are seamlessly integrated into the patient's medical history. Patients have access to their medical records, allowing them to view case papers, diagnoses, prescribed medications, and lab test reports.
    
    The HealthFlow Connect project not only streamlines the patient management process but also enhances accessibility, accuracy, and efficiency in healthcare delivery. This abstract provides an overview of the project's objectives, functionalities, and technologies employed, highlighting its significance in modernizing healthcare systems.
\end{abstract}